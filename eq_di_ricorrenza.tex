\documentclass[a4paper, 12pt]{article} % Font size (can be 10pt, 11pt or 12pt) and paper size (remove a4paper for US letter paper)

\usepackage[protrusion=true,expansion=true]{microtype} % Better typography
\usepackage{graphicx} % Required for including pictures
\usepackage{wrapfig} % Allows in-line images
\usepackage{enumitem} %%Enables control over enumerate and itemize environments
\usepackage{setspace}
\usepackage{amssymb, amsmath, mathrsfs} %%Math packages
\usepackage{stmaryrd}
\usepackage{mathtools}
\usepackage{mathpazo} % Use the Palatino font
\usepackage[T1]{fontenc} % Required for accented characters
\usepackage{multicol}
\usepackage{array}
\usepackage{bibentry}
\usepackage[round]{natbib} %%Or change 'round' to 'square' for square backers
\usepackage[margin = 1in]{geometry}
\setcitestyle{aysep={}}

\linespread{1.05} % Change line spacing here, Palatino benefits from a slight increase by default

\newcommand{\corner}[1]{\ulcorner#1\urcorner} %%Corner quotes
\newcommand{\tuple}[1]{\langle#1\rangle} %%Angle brackets
\newcommand{\set}[1]{\lbrace#1\rbrace} %%Set brackets
\newcommand{\interpret}[1]{\llbracket#1\rrbracket} %%Double brackets
%\DeclarePairedDelimiter\ceil{\lceil}{\rceil}    

\makeatletter
\renewcommand\@biblabel[1]{\textbf{#1.}} % Change the square brackets for each bibliography item from '[1]' to '1.'
\renewcommand{\@listI}{\itemsep=0pt} % Reduce the space between items in the itemize and enumerate environments and the bibliography

\renewcommand{\maketitle}{ % Customize the title - do not edit title and author name here, see the TITLE block below
\begin{flushright} % Right align
{\LARGE\@title} % Increase the font size of the title

\vspace{10pt} % Some vertical space between the title and author name

{\@author} % Author name
\\\@date % Date

\vspace{30pt} % Some vertical space between the author block and abstract
\end{flushright}
}

%----------------------------------------------------------------------------------------
%	TITLE
%----------------------------------------------------------------------------------------

\title{\textbf{Equazioni di Ricorrenza}} % Subtitle

\author{\textsc{Algoritmi 1}\\ \em David Dragomir} % Institution

\date{\today} % Date

%----------------------------------------------------------------------------------------

\begin{document}

\maketitle % Print the title section

\thispagestyle{empty}

%----------------------------------------------------------------------------------------

\section*{Metodo Iterativo}

Si "srotola" la ricorsione, ottenendo una sommatoria dipendente dalla
sola dimensione di $n$. \\

Considerando la seguente equazione di ricorrenza:

$$
T(n) = \begin{cases}
  \theta(1) & \text{se $n=0$} \\
  T(\frac{n}{2}) + \theta(1) & \text{se $n \geq 1$}
\end{cases}
$$

si avrà:

\begin{gather*}
T(n) = T(\frac{n}{2}) + \theta(1) = T(\frac{n}{4}) + \theta(1) + \theta(1) = 
T(\frac{n}{8}) + \theta(1) + \theta(1) + \theta(1) \\
= T(\frac{n}{2^{3}}) + 3 \cdot \theta(1)
\end{gather*}

ci fermiamo quando $\frac{n}{2^{i}} = 1$, dunque:

\begin{gather*}
n = 2^{i} \Rightarrow i = log_{2} n \\
  \Rightarrow log_{2}n \cdot \theta(1) + T(1) \Rightarrow log_{2}n
  \cdot \theta(1) + \theta(1) \simeq \theta(log_{2} n)
\end{gather*}

Altro esempio:
$$
T(n) = \begin{cases}
  n + T(\frac{n}{2}) & \text{se $n > 1$} \\
  1 & \text{se $n = 1$} 
\end{cases}
$$

Sviluppando il termine $n + T(\frac{n}{2})$:

\begin{gather*}
  T(n) = n + T(\frac{n}{2}) = n + \frac{n}{2} + T(\frac{n}{4}) =
  n + \frac{n}{2} + \frac{n}{4} + T(\frac{n}{8}) \\
  = \frac{n}{2^{0}} + \frac{n}{2^{1}} + \frac{n}{2^{2}} + T(\frac{n}{2^{3}}) = \sum_{i = 0}^{k - 1} \frac{n}{2^{i}} + T(\frac{n}{2^k})
\end{gather*}

dunque ci fermeremo quando $\frac{n}{2^k} = 1$:

$$
n = 2^{k} \Rightarrow k = log_{2} n  \\
$$

la sommatoria quindi diventerà:
\begin{gather*}
  \sum_{i = 0}^{log_{2}n - 1} \frac{n}{2^i} + T(1) \\
  \text{conoscendo la sommatoria notevole: $\sum_{i = 0}^{n} \alpha^{i} = \frac{1-\alpha^{n+1}}{1 - \alpha}$} \\
  \Downarrow \\
  n \sum_{i = 0}^{log_{2}n-1} (\frac{1}{2})^{i} + T(1) \Rightarrow n \cdot \frac{1 - (\frac{1}{2})^{log_{2}n}}{1 - (\frac{1}{2})} + 1 \\
  = 2n \cdot (1 - \frac{1}{2^{log_{2}n}}) + 1 = 2n \cdot (1 - \frac{1}{n}) + 1 = 2n \cdot (\frac{n - 1}{n}) + 1 \\
  = 2n - 2 + 1 = 2n - 1 = \theta(n)
\end{gather*} 

\vfill

\bibliographystyle{Phil_Review} %%bib style found in bst folder, in bibtex folder, in texmf folder.
\bibliography{Zotero} %%bib database found in bib folder, in bibtex folder
\end{document}
